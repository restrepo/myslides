\section{$\Lambda$CDM paradigm (with baryonic effects)}



% \begin{frame}
%   \begin{quote}
%     \from{\url{https://phys.org/news/2018-06-years-scientists-account-universe.html}} 
% Dark matter connects clusters of galaxies with massive tendrils, forming a cosmic web that serves as an unseen skeleton for the universe.
%   \end{quote}
% Read more at: 

%   \frametitle{General considerations}
%   \note{From doi:  [10.1098/rsta.2009.0209]
%     We live in an exciting, if somewhat bewildering, time in terms of our understanding of the Universe. The concept of dark matter, as a significant component of the Universe, dates back to the 1970s following the realization that most galaxies are surrounded by unseen haloes. The null results of the Galactic microlensing surveys and the abundance of light elements produced in the big bang further suggest that dark matter is non-baryonic in form. Structure formation models can account for the presently observed large-scale distribution of galaxies if this non-baryonic matter is made of massive non-relativistic (i.e. ‘cold’) particles. Despite this progress, after 40 years, we have yet to detect the dark matter particle itself. Given how both strong and weak gravitational lensing have, in barely a decade, considerably added to our knowledge of the distribution and amount of dark matter, it is relevant to ask what these techniques can offer in the future.}

%   \note{Astronomical observations can define the spatial distribution of dark matter from small to large scales. Theory predicts this distribution in terms of the spectrum of physical scales over which the density of dark matter fluctuates. }
% \end{frame}

\begin{frame}
      \begin{figure}
    \centering
    \includegraphics[scale=0.4]{planck2018}\\
     \tiny{Credit: Planck 2018}
  \end{figure}

\small 
  Why was the temperature of the CMB the same in all directions? \\
  What was the origin of the small temperature fluctions?
  
\end{frame}



\begin{frame}
      \begin{figure}
    \centering
    \only<1>{\includegraphics[scale=0.6]{cmbsoup0}}%
    \only<2>{\includegraphics[scale=0.6]{cmbsoup1}}\\
     %\tiny{Credit: Planck 2018}
  \end{figure}
  
\end{frame}



\begin{frame}
  \begin{figure}
    \centering
    \def\sdm{0.4}
    \only<1>{\includegraphics[scale=\sdm]{soup}}%
    \only<2>{\includegraphics[scale=\sdm]{soup1}}\\
    \only<1->{\tiny Credit: Komatsu, ICTP Summer School on Cosmology 2018\footnote{\tiny Video avalaible}}
  \end{figure}

\end{frame}

\begin{frame}
  \frametitle{Dark matter simulations}
  \begin{figure}
    \centering
    \includegraphics[scale=0.4]{dmtypes}\\
    {\tiny Credit: Arianna Di Cintio (Conference on Shedding Light on the Dark Universe with Extremely Large Telescopes, ICTP - 2018)}
  \end{figure}
\end{frame}

\begin{frame}
  \frametitle{Baryonic effects}
  \begin{figure}
    \centering
    \includegraphics[scale=0.4]{snbang}
  \end{figure}
  Once the effect of baryonic physics is included, it is
  hard to distinguish between WDM/SIDM/CDM

  \begin{center}
    	
\footnotesize See: Gravitational probes of dark matter physics, M.R. Buckley, A.H.G. Peter, arxiv:1712.06615 [PR]
  \end{center}
\end{frame}

\begin{frame}
  \frametitle{Goal}
  \begin{columns}
  \begin{column}{0.65\textwidth}
  \includegraphics[scale=0.2]{galax}     
  \end{column}
  \begin{column}{0.25\textwidth}
    The DESI experiment\\
    \includegraphics[scale=0.15]{DESI}
    {\tiny Credits: J. Forero \url{http://cosmology.univalle.edu.co/} }
  \end{column}
\end{columns}

  %\vspace{-0.3cm}

\end{frame}

\begin{frame}
  \frametitle{Cooking the soup: Cosmic web}
      Dark matter in the universe evolves through gravity to form a complex network
of halos, filaments, sheets and voids, that is known as the cosmic web [arXiv:1801.09070]

An excess of a gas is observed between Milky Way and Andromeda

  \begin{figure}
    \centering
    \includegraphics[scale=0.2]{cosmic-web}\\
    {\tiny Milenium simulation: \url{https://wwwmpa.mpa-garching.mpg.de/galform/virgo/millennium/}}
  \end{figure}
\end{frame}



\begin{frame}
  \frametitle{Cosmic Anatomy}
  \def\sana{0.8cm}
  \centering
  { \tiny Baryons \hspace{\sana} Missing Baryons \hspace{\sana} Dark Matter}
  \begin{figure}
    \centering
    \includegraphics[scale=0.6]{anatomy}\\
  \end{figure}
\end{frame}


\begin{frame}
    \frametitle{The muscles}
\begin{figure}
  \centering
  \includegraphics[scale=0.5]{muscles-anatomy-back}\\
  {\tiny Credit: \url{http://sciencedrivennutrition.com}}
\end{figure}
\end{frame}

\begin{frame}
  \frametitle{Direct observations of filaments}
  \small
  \textbf{Where are the Baryons?} (Cen, Ostriker, astro-ph/9806281 [AJ])
  \begin{quote}
   \scriptsize Thus, not only is the universe dominated by dark matter, but more than one half of the normal matter is yet to be detected. (the muscles)
  \end{quote}

  \vspace{-0.5cm}

\begin{columns}
  \begin{column}{0.48\textwidth}
\begin{figure}
  \centering
  \includegraphics[scale=0.24]{clusterboXtemp}\\
  {\tiny Credit: Cen, arXiv:1112.4527 [AJ]}
\end{figure}

  \end{column}
  \begin{column}{0.48\textwidth}
    \scriptsize
    Warm-hot intergalactic medium (WHIM)

    Density-weighted temperature projection of a portion of the refinement box of the C run of size  $\left(18\,h^{-1}\text{Mpc} \right)^3$.

    Low temperature WHIM confirmed by O VI line that peak at $T\sim 3\times 10^5\ \text{K}$
  \end{column}
\end{columns}

\small   
\includegraphics[scale=0.1]{new}Hotter phases of the WHIM: \textbf{  Observations of the missing baryons in the warm-hot intergalactic medium}
  (Nicastro, \emph{et al.} arXiv:1806.08395 [Nature]). 

\end{frame}
  % umerical simulations in the framework of the commonly accepted (ΛCDM) cosmological paradigm predict that, starting at a redshift of $z\approx 2$ and during the continuous process of structure formation, diffuse baryons in the intergalactic medium (IGM) condense into a filamen-tary web (with electron densities of $n_e\approx 10^{-6}-10^{-4}\ \text{cm}^{-3}$ and undergo shocks that heat them up to temperatures of $T\approx 10^5-10^7\ \text{K}$, making the by-far-largest constituent of the IGM, hydrogen, mostly ionized. At the same time, galactic outflows powered by stellar and active galactic nucleus (AGN) feedback enrich the IGM baryons with metal. How far from galaxies these metals roam depends on the energetics of these winds, but it is expected that metals and galaxies are spatially correlated.
  
  %     \begin{quote}
  % From: \url{https://phys.org/news/2018-06-years-scientists-account-universe.html}

  %   The scientists used the European Space Agency's XMM-Newton X-ray space telescope to study the BL Lacertae quasar 1ES 1553+113, an active, supermassive black hole at the center of a galaxy. Quasars gobble up matter and shine brightly in many wavelengths of light, from radio waves to X-rays. These celestial lighthouses can basically back light the material that crosses the beam's path, just as a flashlight beam illuminates unseen motes of dust in the air.
  % \end{quote}

  % \begin{quote}

  %   Studying the chemical fingerprint of oxygen in the X-rays from the quasar light, the scientists found a large amount of extremely hot intergalactic gas so much that they calculate that this gas could account for up to 40 percent of the baryonic matter in the cosmos, which could be enough to explain the missing matter.

  % \end{quote}

  % \begin{quote}

  %   Taotao Fang of the Jiujiang Research Institute in China, who was not involved in the study, pointed to a few possible alternate explanations, including that the ionized gas signal may have come from within a galaxy rather than from intergalactic gas embedded in a dark matter filament.
  % \end{quote}
  



\begin{frame}
    \frametitle{The skeleton}
\begin{figure}
  \centering
  \includegraphics[scale=0.5]{xray}\\
  {\tiny Credit: \url{https://www.disnola.com}}
\end{figure}
\end{frame}







% From: \url{http://www2.cnrs.fr/en/1247.htm},  
% \begin{quote}
%   the MegaCam, which has 340 megapixels, using a technique known as  “This technique works somewhat like an X-ray,” says Kilbinger. “You can’t see inside the human body, and we can’t see the dark matter directly.” Dark matter is dark, but any light emitted from galaxies must pass around it, and the diversion is recorded as a gravitational distortion. Astronomers can then use this gravitational imprint to calculate the size of the matter that caused it. Using such techniques, astronomers estimate that dark matter forms web-like structures, made up of clusters, filaments, and large rope-like structures that the new research has uncovered. But only until dark matter becomes visible will these assumptions become fact. Nonetheless, the new findings open up a dizzying vista of possibility, where, with better telescopes in the future, more precise lay-out and details of dark matter could be revealed. Understanding the formation of dark matter could help scientists improve their knowledge on how the universe evolved, and what its future might be. “There are other techniques that can measure dark matter,” says Kilbinger, “but over the next 10 to 20 years, I think this one will become the most used. And that’s why people are so excited about it.”
% \end{quote}


% From:
% \begin{quote}
%   Looking deep into space, and literally peering back in time, is like experiencing the universe in a house of mirrors where everything is distorted through a phenomenon called gravitational lensing. 
% \end{quote}



% coupled with the angular resolution of ELTs this will open a unique window to constrain the
% dark matter properties with detail and statistical completeness. 










% \begin{frame}
%   \frametitle{Evidences at all redshifts.}
% \begin{columns}
%   \begin{column}{0.48\textwidth}
%       \begin{itemize}
%   \item   For cluster of Galaxies
%   \item For stars inside Galaxies
%   \item Inside our Galaxy
%   \item Senatore reconstruction
%   \item En las simulaciones se parte de la CMB
%   \end{itemize}
%   \end{column}
%   \begin{column}{0.48\textwidth}

%       \begin{figure}
%     \centering
%     \includegraphics[scale=0.2]{gallightcrv}
%   \end{figure}

%   \end{column}
% \end{columns}

% \end{frame}



% \begin{frame}
%   \frametitle{Inner Milky Way}
%   \url{https://www.nature.com/articles/nphys3237}
% \end{frame}

